\section{Kinematik}
Allgemein kann folgende Umrechnung für die Zeit verwendet werden:
\[1 \frac{m}{s} \eqi 3.6 \frac{km}{h}\]

\subsection{Lineare Bewegung}
Konstante Geschwindigkeit
\begin{formula}
	{s = v \cdot t} 
	
	s & Strecke & [km] \\
	v & Geschwindigkeit & [km/h] \\
	t & Zeit & [h]
\end{formula}

Konstante Beschleunigung
\begin{formula}
	{s(t) = \frac{at^2}{2} + v_0t + s_0} 
	
	a & Beschleunigung & [m] \\
	v_0 & Anfangs Geschw. & [m/s] \\
	s_0 & Anfangs Strecke & [m]
\end{formula}

\begin{formula}
	{v_e = \sqrt{v_0^2 + 2as}} 
	
	a & Beschleunigung & [m] \\
	v_0 & Anfangs Geschw. & [m/s] \\
	v_e & End Geschw. & [m/s] \\
	s & Strecke & [m]
\end{formula}

\subsection{Kreisbewegung}
\todo{}


\subsection{Wurf}
\subsubsection{Senkrechter Wurf}
\todo{}

\subsubsection{Horizontaler Wurf}
\todo{}

\subsubsection{Schiefer Wurf}
\todo{}

\subsection{Pendel}
\todo{}
