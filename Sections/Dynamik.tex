
\section{Dynamik}
\subsection{Kreisbewegung}\label{kreisbewegung}
\noindent Konstante Geschwindigkeit
\begin{formula}
	{\phi = \omega \cdot t} 
	
	\phi & Winkelweg & [rad] \\
	\omega & Winkelgeschwindigkeit & [1/s] \\
	t & Zeit & [s]
\end{formula}
\begin{formula}
	{\omega = 2\pi \cdot u} 
	
	\omega & Winkelgeschwindigkeit & [1/s] \\
	u & Umdrehungen & [Hz, s^{-1}] 
\end{formula}

\noindent Konstante Beschleunigung
\begin{formula}
	{\phi(t) = \frac{\alpha t^2}{2} + \omega_0 t + \phi_0 
		\\
		\omega = \alpha \cdot t} 
	
	\alpha & Winkelbesch. & [1/s^2] \\
	\omega_0 & Anfangs W.Geschw. & [1/s] \\
	\phi_0 & Anfangs W.Weg & [rad]
\end{formula}

\noindent Ohne Anfangsgeschwindigkeit
\begin{formula}
	{\omega = \sqrt{2 \cdot \alpha \cdot \phi}} 
	
	\alpha & Winkelbesch. & [1/s^2] \\
	\omega & W.Geschw. & [1/s] \\
	\phi & W.Weg & [rad]
\end{formula}

\noindent Mit Anfangsgeschwindigkeit
\begin{formula}
	{\omega_e = \sqrt{\omega_0^2 + 2\alpha \phi}} 
	
	\alpha & Winkelbesch. & [1/s^2] \\
	\omega_e & End W.Geschw. & [1/s] \\
	\omega_0 & Anfangs W.Geschw. & [1/s] \\
	\phi & W.Weg & [rad]
\end{formula}

\noindent Die Winkelgeschwindigkeit Funktion kann verwendet werden, um die Dauer eines Abbremsvorgangs zu berechnen, da $\omega(t) = 0$ sein muss beim Stillstand und somit gilt: $t = \frac{-\omega_0}{\alpha}$
\begin{formula}
	{\omega(t) = \alpha \cdot t + \omega_0} 
	
	\alpha & Winkelbesch. & [1/s^2] \\
	\omega_0 & Anfangs W.Geschw. & [1/s] \\
\end{formula}

\subsubsection{Umrechnung Linear- zu Kreis-Bewegung}
Von einer Linearen in eine Kreisbewegung umrechnen (Siehe auch \verweiseref{kreisbewegung} und \verweiseref{linearbewegung}):
\[\alpha = \frac{a}{r} \qquad \phi = \frac{s}{r} \qquad \omega = \frac{v}{r}\]

\subsection{Rotation}
\begin{formula}
	{M = J \cdot \alpha} 
	
	M & Drehmoment & [Nm] \\
	J & Trägheitsmoment Achse & [kg \cdot m^2] \\
	\alpha & Winkelbesch. & [1/s^2]
\end{formula}
\begin{formula}
	{u = \frac{\omega}{2\pi} = \frac{\alpha \cdot t}{2\pi}} 
	
	u & Umdrehungen & [s^{-1}] \\
	\omega & W.Geschw. & [1/s] \\
	\alpha & Winkelbesch. & [1/s^2]
\end{formula}

\subsection{Drehimpuls \kuchling{138}}
\begin{formula}
	{L = M \cdot t =\\ J \cdot \omega = \vec{r} \times \vec{p} \\} 
		
	L & Drehimpuls & [\frac{kg\cdot m^2}{s}, Nms] \\
	M & Drehmoment & [Nm] \\
	\omega & Winkelgeschwindigkeit & [1/s] \\
	J & Trägheitsmoment Achse & [kg \cdot m^2]
\end{formula}

\subsection{Trägheitsmomente}
Siehe \kuchling{131} für Tabelle!
