\section{Impuls \kuchling{119}}
Ein Impuls ist folgendermassen beschrieben:
\begin{formula}
	{F = \frac{p}{t} \rightarrow p = m \cdot v} 
	
	F & Kraft & [N] \\
	p & Impuls & [Ns, kg \cdot m/s] \\
	t & Dauer der Kraft & [s] \\
	m & Masse & [kg] \\
	v & Geschw. & [m/s]
\end{formula}

\noindent Der Impulserhaltungs-Satz besagt:
\[
m_1 \cdot v_1 + m_2 \cdot v_2 = m_1 \cdot v_1' + m_2 \cdot v_2'
\]

\noindent \textbf{Elastischer Stoss}
\[
v_1' = \frac{(m_1 - m_2)v_1 + 2m_2v_2}{m_1 + m_2}
\]

\noindent \textbf{Unelastischer Stoss}
\begin{formula}
	{W = E_1 - E_2 =\\ \frac{m_1 \cdot m_2}{2(m_1 + m_2)}(v_1 - v_2)^2} 
	
	W & Deformationsarbeit & [J, Ws] \\
	E_1 & $\sum E_{Kin}$ \underline{vor} Stoss & [J] \\
	E_2 & $\sum E_{Kin}$ \underline{nach} Stoss & [J] \\
\end{formula}

\todo{Schwerpunktgeschwindigkeit\\
	\[
		u = \frac{m_1v_1 + m_2v_2}{m_1 + m_2}
	\]
	\[
		E_{Kin} = \frac{m_1 + m_2}{2}u^2 + \frac{m_1}{2}(v_1 - u)^2 + \frac{m_2}{2}(v_2 - u)^2
	\]
}


\subsection{Drehimpuls \kuchling{138}}
\begin{formula}
	{L = M \cdot t =\\ J \cdot \omega = \vec{r} \times \vec{p}} 
		
	L & Drehimpuls & [\frac{kg\cdot m^2}{s}, Nms] \\
	M & Drehmoment & [Nm] \\
	\omega & Winkelgeschwindigkeit & [1/s] \\
	J & Trägheitsmoment Achse & [kg \cdot m^2]
\end{formula}
\\
\includegraphics[width=\columnwidth]{./Images/Trägheitsmoment.png}