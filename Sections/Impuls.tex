\section{Impuls \kuchling{119}}
Ein Impuls ist folgendermassen beschrieben:
\begin{formula}
	{F = \frac{\Delta p}{\Delta t} \rightarrow p = m \cdot v} 
	
	F & Kraft & [N] \\
	p & Impuls & [Ns, kg \cdot m/s] \\
	t & Dauer der Kraft & [s] \\
	m & Masse & [kg] \\
	v & Geschw. & [m/s]
\end{formula}

\noindent Der Impulserhaltungs-Satz besagt:
\begin{formula}
	{m_1 \cdot v_1 + m_2 \cdot v_2 \\= m_1 \cdot v_1' + m_2 \cdot v_2'} 
	
	m & Masse & [kg] \\
	v & Geschw. \underline{vor} Stoss & [m/s] \\
	v' & Geschw. \underline{nach} Stoss & [m/s] \\
\end{formula}

\subsection{Elastizität}
\includegraphics[width=\columnwidth]{./Images/elastizität.png}

\subsection{Elastischer Stoss} \kuchling{119}
\begin{formula}
	{v_1' = \frac{(m_1 - m_2)v_1 + 2m_2v_2}{m_1 + m_2} \\ \\
 	 v_2' = \frac{(m_2 - m_1)v_2 + 2m_1v_1}{m_2 + m_1}}
  
  	m & Masse & [kg] \\
	v & Geschw. \underline{vor} Stoss & [m/s] \\
    v' & Geschw. \underline{nach} Stoss & [m/s] \\
\end{formula}

\subsection{Unelastischer Stoss} \kuchling{121}
\begin{formula}
	{v' = \frac{m_1v_1 + m_2v_2}{m_1 + m_2}}
	
	m & Masse & [kg] \\
	v & Geschw. \underline{vor} Stoss & [m/s] \\
	v' & Geschw. \underline{nach} Stoss & [m/s] \\
\end{formula}

\noindent Verformungsarbeit\label{verformungsarbeit}
\begin{formula}
	{W &= E_1 - E_2 \\ &= \frac{m_1 \cdot m_2}{2(m_1 + m_2)}(v_1 - v_2)^2} 
	
	W & Deformationsarbeit & [J, Ws] \\
	E_1 & $\sum E_{Kin}$ \underline{vor} Stoss & [J] \\
	E_2 & $\sum E_{Kin}$ \underline{nach} Stoss & [J] \\
\end{formula}

\subsubsection{Schwerpunktgeschwindigkeit}
Die Schwerpunktgeschwindigkeit $u$ kann mit folgender Formel berechnet werden. Damit können System, welche sich relativ bewegen, mit einem fixen Bezugspunkt berechnet werden.
\[
	u = \frac{m_1v_1 + m_2v_2}{m_1 + m_2}
\]
\[
	E_{Kin} = \frac{m_1 + m_2}{2}u^2 + \frac{m_1}{2}(v_1 - u)^2 + \frac{m_2}{2}(v_2 - u)^2
\]

\subsection{Dehnung \kuchling{184}}
\begin{formula}
	{\sigma = \frac{F}{A} \\ = E \frac{\Delta l}{l}	= E\varepsilon}
	
	\varepsilon & Relative Längenänd. & [1]\\
	l & Länge \underline{ohne} Kraft & [m]\\
	\Delta l & Längeänd. \underline{mit} Kraft & [m]\\
	E & Elastizitätmodul & \\
	\sigma & Druck & [N/m^2]\\
	A & Querschnittsfläche & [m^2]\\
\end{formula}


\newpage