\section{Energie}
Energie ist gespeicherte Arbeit und kann weder zerstört noch erschaffen werden. Energieerhaltungs-Satz!

\begin{formula}
	{E = P \cdot t} 
	
	E & Energie & [J, Ws, Nm] \\
	P & Leistung & [W] \\
	t & Zeit & [s]
\end{formula}

\noindent Es gibt verschiedene Arten von Energie welche ineinander umgewandelt werden können. Die Summe bleibt jedoch gleich $E_{Total} = \sum E_x$:
\begin{align*}
	E_{Kin} &= \frac{mv^2}{2} &\quad E_{Pot} &= m \cdot g \cdot h \\
	E_{Feder} &= \frac{x}{2} \cdot \Delta s^2 &\quad E_{Reib} &= F_R \cdot \Delta s \\
	E_{Rot}  &= \frac{J}{2}\omega^2
\end{align*}
\begin{formula}{}	
	x & Federkonstante & [N/m] \\
	\omega & Winkel-Geschw. & [1/s, Umdr./s] \\
	J & Trägheitsmoment & [kg \cdot m^2]
\end{formula}

\section{Leistung}
\begin{formula}
	{P = \frac{W}{t} = F \cdot v = \frac{E}{t}} 
	
	W & Arbeit & [Nm, J, Ws] \\
	F & Kraft & [N] \\
	v & Geschw. & [m/s] \\
	t & Zeit & [s] \\
	E & Energie & [J, Ws, Nm] \\
\end{formula}

\noindent Der Wirkungsgrad eines Systems kann mit $n = \frac{P_{ab}}{P_{Auf}}$ berechnet werden.
