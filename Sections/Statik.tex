\section{Statik}
Generelles Vorgehen:
\begin{enumerate}[nosep]
	\item Skizze mit allen Kräften
	\item Koordinatensystem einführen
	\item Drehpunk von Drehmoment einführen
	\item Gleichgewichtsbedingung und Gleichungssystem aufstellen
	\item Gleichungssystem lösen
\end{enumerate}

~\\
\textbf{Beispiel Skizzen}

\begin{minipage}{\textwidth}	
	\begin{minipage}{0.25\textwidth}
		Haften:\\
		\includegraphics[width=\columnwidth]{./Images/Haften.png}
	\end{minipage}%%% to prevent a space
	\begin{minipage}{0.25\textwidth}
		Kippen\\
		\includegraphics[width=\columnwidth]{./Images/Kippen.png}
	\end{minipage}
\end{minipage}



\subsection{Reibung}
\begin{formula}
	{F_R = F_N \cdot \mu} 
	
	F_R & Reibungskraft & [N] \\
	F_N & Normalkraft & [N] \\
	\mu & Koeffizient & [1]
\end{formula}
\todo{was ist a ?}
\noindent Der Koeffizient ist $\mu_H = \tan(\alpha)$ oder auf einer Schiefe Ebene $\mu_G= \tan(\alpha) - \frac{a}{g\cdot\cos(\alpha)}$.

\subsection{Dichte}
\begin{formula}
	{\rho = \frac{m}{V}} 
	
	\rho & Dichte & [kg/m^3] \\
	m &    Druck & [N/m^2] \\
	V &    Volumen & [m^3]
\end{formula}

\subsection{Dehnung}
\todo{}
\begin{formula}
	{\varepsilon = \frac{\Delta l}{l} = \frac{1}{E} \cdot \sigma} 
	
	\varepsilon & Dehnung & [m]\\
	l & Länge & [m]\\
	E & Elastizitätmodul & \\
	\sigma & Druck & [N/m^2]\\
\end{formula}


\subsection{Druck/Spannung}
\todo{}
\begin{formula}
	{\sigma = \frac{F_\perp}{A}} 
	
	\sigma & Druck & [N/m^2]\\
	...
\end{formula}
\begin{formula}
	{\tau = \frac{F_\parallel}{A}} 
	
	\tau & Schubspannung & [N/m^2]\\
	...
\end{formula}